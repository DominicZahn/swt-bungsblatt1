\documentclass[parskip=full]{scrartcl}
\usepackage[utf8]{inputenc}
\usepackage[german]{babel}
\usepackage{hyperref}
\hypersetup{
    pdftitle={iMage App: Lastenheft},
    bookmarks=true,
}
\usepackage{graphicx}
\usepackage{csquotes}
\usepackage[nonumberlist]{glossaries}
\usepackage{enumitem}

\makenoidxglossaries

\newglossaryentry{App}
{
    name=App,
    plural=Apps,
    description={Das Wort App kommt aus dem Englischen und ist die Kurzform des Wortes Application.
    Meist wird es jedoch als Synonym für ein Handyprogramm verwendet.}
}

\newglossaryentry{HDR-Bild}{
    name=HDR-Bild,
    plural=HDR-Bilder,
    description={Ein HDR-Bild ist ein Bild, welches ein großes Spektrum an Kontrasten abdeckt und 
    somit beispielsweise sehr starke Helligkeitsunterschiede abbilden kann.
    HDR ist eine Abkürzung für High Dynamic Range.}
}

\title{iMage App: Lastenheft}
\author{Dominic Zahn}

\begin{document}

\maketitle

\section{Zielbestimmung}

Die \gls{App} iMage dient der Erstellung von \gls{HDR-Bild}. Sie soll dem Benutzer eine einfache und
schnelle Möglichkeit bieten ein eigenes Bild in ein \gls{HDR-Bild} zu transformieren.

\section{Produkteinsatz}

Das Produkt dient dazu Bilder, welche nicht im HDR-Format aufgenommen wurden, in dieses zu überführen.

Zielgruppe: Personen, die keine HDR-Kamera besitzen

Plattform: Android 6 oder Nachfolger, ios 8 oder Nachfolger



\printnoidxglossaries

\end{document}