\documentclass[parskip=full]{scrartcl}
\usepackage[utf8]{inputenc}
\usepackage[german]{babel}
\usepackage{hyperref}
\hypersetup{
    pdftitle={iMage App: Lastenheft},
    bookmarks=true,
}
\usepackage{graphicx}
\usepackage{csquotes}
\usepackage[nonumberlist]{glossaries}
\usepackage{enumitem}

\makenoidxglossaries

\newglossaryentry{App}
{
    name=App,
    plural=Apps,
    description={Das Wort App kommt aus dem Englischen und ist die Kurzform des Wortes Application.
    Meist wird es jedoch als Synonym für ein Handyprogramm verwendet.}
}

\newglossaryentry{HDR-Bild}{
    name=HDR-Bild,
    plural=HDR-Bilder,
    description={Ein HDR-Bild ist ein Bild, welches ein großes Spektrum an Kontrasten abdeckt und 
    somit beispielsweise sehr starke Helligkeitsunterschiede abbilden kann.
    HDR ist eine Abkürzung für High Dynamic Range.}
}

\newglossaryentry{Systemspeicher}
{
    name=Systemspeicher,
    description={Als Systemspeicher wird meist der integrierte Speicher eines Mobiltelefons, Computers, etc. bezeichnet.}
}

\newglossaryentry{sozials Netzwerk}
{
    name=soziales Netzwerk,
    plural=soziale Netzwerke,
    description={Dies ist ein Sammelbegriff für Plattformen, die es Nutzern ermöglichen Bilder, Texte oder andere Medien
    mit jeder beliebigen anderen in Person in diesem Netzwerk zu teilen.}
}

\title{iMage App: Lastenheft}
\author{Dominic Zahn}

\begin{document}

\maketitle

\section{Zielbestimmung}
Die \gls{App} iMage dient der Erstellung von \gls{HDR-Bild}. Sie soll dem Benutzer eine einfache und
schnelle Möglichkeit bieten eigene Bilder in ein \gls{HDR-Bild} zu transformieren.

\section{Produkteinsatz}
Das Produkt dient dazu Bilder, welche nicht im HDR-Format aufgenommen wurden, in dieses zu überführen.

Zielgruppe: Besitzer eines Mobiltelefons ohne HDR-Kamera

Plattform: Android 6 oder Nachfolger, ios 8 oder Nachfolger

\section{Funktionale Anforderungen}
\begin{itemize}[nosep]

\item[FA10] Bei Abonnement eines Kunden für den Zeitraum des Abos keine Kaufoptionen anzeigen.
\item[FA20] Die Startseite der Applikation soll eine Übersicht über das Nutzerkonto und die zuletzt erstellten HDRBilder
bereitstellen.
\item[FA30] Anzeigen der Bilder im \gls{Systemspeicher}.
\item[FA40] Der Nutzer kann aus drei Bildern ein \gls{HDR-Bild} erstellen.
\item[FA50] Nach der Umwandlung soll es möglich sein das Ergebinsbild über \glspl{sozials Netzwerk} wie Facezine oder Instagrim
zu teilen.

\end{itemize}

\section{Produktdaten}
\begin{itemize}[nosep]
\item[PD10] HDR-Bilder nach Umwandlung im Speicher des Mobiltelefons ablegen.
\item[PD20] Speicherung der Bilder auf dem Pear-Corp-Zentralserver.
\item[PD30] Speicherung der Benutzerdaten auf dem Pear-Corp-Zentralserver.
    
\end{itemize}

\section{Nichtfuntkionale Anforderungen}
\begin{itemize}[nosep]
\item[NF10] In der Vorschauansicht soll in Echtzeit bis zu einhundert (100)
Bilder auf einem Mittelklasse-Smartphone anzeigen können.
\item[NF20] Die Funktion /FA40/ darf nicht mehr als sieben Sekunden in Anspruch nehmen.
    
\end{itemize}
\printnoidxglossaries

\end{document}